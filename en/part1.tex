\documentclass{beamer}
\input{preamble.tex}

\subtitle{Extended Learning}

\begin{document}

%%%%%%%%%%%%%%%%%%%%%%%%%%%%%%%%%%%%%%%%%%%%%%%%%%%%%%%%%%%%%%%%%%%%%%%%%%%%%%%
%%%%%%%%%%%%%%%%%%%%%%%%%%%%%%%%%%%%%%%%%%%%%%%%%%%%%%%%%%%%%%%%%%%%%%%%%%%%%%%
%%%%%%%%%%%%%%%%%%%%%%%%%%%%%%%%%%%%%%%%%%%%%%%%%%%%%%%%%%%%%%%%%%%%%%%%%%%%%%%
\begin{frame}
\titlepage
\end{frame}

%%%%%%%%%%%%%%%%%%%%%%%%%%%%%%%%%%%%%%%%%%%%%%%%%%%%%%%%%%%%%%%%%%%%%%%%%%%%%%%
%%%%%%%%%%%%%%%%%%%%%%%%%%%%%%%%%%%%%%%%%%%%%%%%%%%%%%%%%%%%%%%%%%%%%%%%%%%%%%%
%%%%%%%%%%%%%%%%%%%%%%%%%%%%%%%%%%%%%%%%%%%%%%%%%%%%%%%%%%%%%%%%%%%%%%%%%%%%%%%
\begin{frame}{Plan}
We are set up with the presentation on one monitor and a live demo on the other, so that you can practice using Overleaf and \LaTeX.

Today we will cover:
\begin{itemize}
\item Adding Text
\item Adding Images
\item Adding Mathematics
\item Formatting the document:
\begin{itemize}
\item Adding title, author, date
\item Creating sections
\item Adding a table of contents
\end{itemize}
\item Adding bibliography
\end{itemize}
\end{frame}
%%%%%%%%%%%%%%%%%%%%%%%%%%%%%%%%%%%%%%%%%%%%%%%%%%%%%%%%%%%%%%%%%%%%%%%%%%%%%%%
%%%%%%%%%%%%%%%%%%%%%%%%%%%%%%%%%%%%%%%%%%%%%%%%%%%%%%%%%%%%%%%%%%%%%%%%%%%%%%%
%%%%%%%%%%%%%%%%%%%%%%%%%%%%%%%%%%%%%%%%%%%%%%%%%%%%%%%%%%%%%%%%%%%%%%%%%%%%%%%
\begin{frame}{Why \LaTeX{}?}
\begin{itemize}
\item It makes beautiful documents
\begin{itemize}
\item Especially mathematics
\end{itemize}
%
\item It was created by scientists, for scientists
\begin{itemize}
\item A large and active community
\end{itemize}
%
\item It is powerful --- you can extend it
\begin{itemize}
\item Packages for papers, presentations, spreadsheets, \ldots
\end{itemize}
%
\item Publishers use templates to ensure standardization of formatting
\begin{itemize}
\item Journals, theses, books, \ldots
\item Some of these templates are even available in Overleaf!
\end{itemize}

\end{itemize}
\end{frame}

%%%%%%%%%%%%%%%%%%%%%%%%%%%%%%%%%%%%%%%%%%%%%%%%%%%%%%%%%%%%%%%%%%%%%%%%%%%%%%%
%%%%%%%%%%%%%%%%%%%%%%%%%%%%%%%%%%%%%%%%%%%%%%%%%%%%%%%%%%%%%%%%%%%%%%%%%%%%%%%
%%%%%%%%%%%%%%%%%%%%%%%%%%%%%%%%%%%%%%%%%%%%%%%%%%%%%%%%%%%%%%%%%%%%%%%%%%%%%%%
\subsection{Getting started}
\begin{frame}[fragile]{\insertsubsection{} with Overleaf}
\begin{itemize}
\item Overleaf is a website for writing documents in \LaTeX.
\item It `compiles' your \LaTeX{} automatically to show you the results.
\vskip 2em
\item Sign in or register for a new account with \wllogo{} 
\\
at \url{https://www.overleaf.com/register}
\item Confirm email address.
\item Enter name, familiarity with \LaTeX, and university information
\item Create a new project - choose blank project
\item Name the new project ``2025 LaTeX Workshop''
\item Download the chignecto files onto your computer.
\item Upload the files to your project.
\end{itemize}
\end{frame}
%%%%%%%%%%%%%%%%%%%%%%%%%%%%%%%%%%%%%%%%%%%%%%%%%%%%%%%%%%%%%%%%%%%%%%%%%%%%%%%
%%%%%%%%%%%%%%%%%%%%%%%%%%%%%%%%%%%%%%%%%%%%%%%%%%%%%%%%%%%%%%%%%%%%%%%%%%%%%%%
%%%%%%%%%%%%%%%%%%%%%%%%%%%%%%%%%%%%%%%%%%%%%%%%%%%%%%%%%%%%%%%%%%%%%%%%%%%%%%%

\begin{frame}[fragile]{How does it work?}
\begin{itemize}
\item You write your document in \texttt{plain text} with \cmd{commands} that
describe its structure and meaning.
\item The \texttt{latex} program processes your text and commands to produce a
beautifully formatted document.
\end{itemize}
\vskip 2ex
\begin{center}
\begin{minted}[frame=single]{latex}
The rain in Spain falls \emph{mainly} on the plain.
\end{minted}
\vskip 2ex
\tikz\node[single arrow,fill=gray,font=\ttfamily\bfseries,%
  rotate=270,xshift=-1em]{latex};
\vskip 2ex
\fbox{The rain in Spain falls \emph{mainly} on the plain.}
\end{center}
\end{frame}

%%%%%%%%%%%%%%%%%%%%%%%%%%%%%%%%%%%%%%%%%%%%%%%%%%%%%%%%%%%%%%%%%%%%%%%%%%%%%%%
%%%%%%%%%%%%%%%%%%%%%%%%%%%%%%%%%%%%%%%%%%%%%%%%%%%%%%%%%%%%%%%%%%%%%%%%%%%%%%%
%%%%%%%%%%%%%%%%%%%%%%%%%%%%%%%%%%%%%%%%%%%%%%%%%%%%%%%%%%%%%%%%%%%%%%%%%%%%%%%
\begin{frame}[fragile]{More examples of commands and their output\ldots}
\begin{exampletwoup}
\begin{itemize}
\item New Brunswick
\item Nova Scotia
\item Canada
\end{itemize}
\end{exampletwoup}
\vskip 2ex
\begin{exampletwoup}
\begin{figure}
\includegraphics{gerbil}
\end{figure}
\end{exampletwoup}
\vskip 2ex
\begin{exampletwoup}
\begin{equation}
\alpha + \beta + 1
\end{equation}
\end{exampletwoup}

\tiny{Image license: \href{https://pixabay.com/en/animal-apple-attractive-beautiful-1239390/}{CC0}}
\end{frame}

%%%%%%%%%%%%%%%%%%%%%%%%%%%%%%%%%%%%%%%%%%%%%%%%%%%%%%%%%%%%%%%%%%%%%%%%%%%%%%%
%%%%%%%%%%%%%%%%%%%%%%%%%%%%%%%%%%%%%%%%%%%%%%%%%%%%%%%%%%%%%%%%%%%%%%%%%%%%%%%
%%%%%%%%%%%%%%%%%%%%%%%%%%%%%%%%%%%%%%%%%%%%%%%%%%%%%%%%%%%%%%%%%%%%%%%%%%%%%%%
\begin{frame}[fragile]{Adjust your thinking}

\begin{itemize}
\item Use commands to describe `what it is', not `how it looks'.
\item Focus on your content.
\item Let \LaTeX{} do its job.
\end{itemize}
\end{frame}

%%%%%%%%%%%%%%%%%%%%%%%%%%%%%%%%%%%%%%%%%%%%%%%%%%%%%%%%%%%%%%%%%%%%%%%%%%%%%%%
%%%%%%%%%%%%%%%%%%%%%%%%%%%%%%%%%%%%%%%%%%%%%%%%%%%%%%%%%%%%%%%%%%%%%%%%%%%%%%%
%%%%%%%%%%%%%%%%%%%%%%%%%%%%%%%%%%%%%%%%%%%%%%%%%%%%%%%%%%%%%%%%%%%%%%%%%%%%%%%
\begin{frame}[fragile]{The Basics}
\begin{itemize}
\item A minimal \LaTeX{} document:
\inputminted[frame=single]{latex}{basics.tex}
\item Commands start with a \emph{backslash} \keystrokebftt{\bs}.
\item Every document starts with a \cmdbs{documentclass} command.
\item The \emph{argument} in curly braces \keystrokebftt{\{} \keystrokebftt{\}} tells \LaTeX{} what kind of document we are creating: an \bftt{article}.
\item A percent sign \keystrokebftt{\%} starts a \emph{comment} --- \LaTeX{}
will ignore the rest of the line.
\end{itemize}
\end{frame}

%%%%%%%%%%%%%%%%%%%%%%%%%%%%%%%%%%%%%%%%%%%%%%%%%%%%%%%%%%%%%%%%%%%%%%%%%%%%%%%
%%%%%%%%%%%%%%%%%%%%%%%%%%%%%%%%%%%%%%%%%%%%%%%%%%%%%%%%%%%%%%%%%%%%%%%%%%%%%%%
%%%%%%%%%%%%%%%%%%%%%%%%%%%%%%%%%%%%%%%%%%%%%%%%%%%%%%%%%%%%%%%%%%%%%%%%%%%%%%%
\subsection{Typesetting Text}
\begin{frame}[fragile]{\insertsubsection{}}
\small
\begin{itemize}
\item Type your text between \cmdbegin{document} and \cmdend{document}.
\item For the most part, you can just type your text normally.
\begin{exampletwouptiny}
Words are separated by one or more
spaces.

Paragraphs are separated by one
or more blank lines.
\end{exampletwouptiny}
\item Space in the source file is collapsed in the output.
\begin{exampletwouptiny}
The   rain       in Spain
falls mainly on the plain.
\end{exampletwouptiny}
\end{itemize}
\end{frame}

%%%%%%%%%%%%%%%%%%%%%%%%%%%%%%%%%%%%%%%%%%%%%%%%%%%%%%%%%%%%%%%%%%%%%%%%%%%%%%%
%%%%%%%%%%%%%%%%%%%%%%%%%%%%%%%%%%%%%%%%%%%%%%%%%%%%%%%%%%%%%%%%%%%%%%%%%%%%%%%
%%%%%%%%%%%%%%%%%%%%%%%%%%%%%%%%%%%%%%%%%%%%%%%%%%%%%%%%%%%%%%%%%%%%%%%%%%%%%%%
\begin{frame}[fragile]{\insertsubsection{}: Tricky bits}
\small
\begin{itemize}
\item Quotation marks are a bit tricky:\\
use a backtick \keystroke{\`{}} on the left and an apostrophe \keystroke{\'{}} on the right.
\begin{exampletwouptiny}
Single quotes: `text'.

Double quotes: ``text''.
\end{exampletwouptiny}

\item Some common characters have special meanings in \LaTeX:\\[1ex]
\begin{tabular}{cl}
\keystrokebftt{\%} & percent sign              \\
\keystrokebftt{\#} & hash (pound / sharp) sign \\
\keystrokebftt{\&} & ampersand                 \\
\keystrokebftt{\$} & dollar sign               \\
\end{tabular}
\item If you just type these, you'll get an error. If you want one to appear in
the output, you have to \emph{escape} it by preceding it with a backslash.
\begin{exampletwoup}
\$\%\&\#!
\end{exampletwoup}
\end{itemize}
\end{frame}

%%%%%%%%%%%%%%%%%%%%%%%%%%%%%%%%%%%%%%%%%%%%%%%%%%%%%%%%%%%%%%%%%%%%%%%%%%%%%%%
%%%%%%%%%%%%%%%%%%%%%%%%%%%%%%%%%%%%%%%%%%%%%%%%%%%%%%%%%%%%%%%%%%%%%%%%%%%%%%%
%%%%%%%%%%%%%%%%%%%%%%%%%%%%%%%%%%%%%%%%%%%%%%%%%%%%%%%%%%%%%%%%%%%%%%%%%%%%%%%
\begin{frame}[fragile]{Handling Errors}
\begin{itemize}
\item \LaTeX{} can get confused when it is trying to compile your document. If
it does, it stops with an error, which you must fix before it will produce
any output.
\item For example, if you misspell \cmdbs{emph} as \cmdbs{meph}, \LaTeX{} will
stop with an ``undefined control sequence'' error, because ``meph'' is not
one of the commands it knows.
\end{itemize}
\begin{block}{Advice on Errors}
\begin{enumerate}
\item Don't panic! Errors happen.
\item Fix them as soon as they arise --- if what you just typed caused an error,
you can start your debugging there.
\item If there are multiple errors, start with the first one --- the cause may
even be above it.
\end{enumerate}
\end{block}
\end{frame}

%%%%%%%%%%%%%%%%%%%%%%%%%%%%%%%%%%%%%%%%%%%%%%%%%%%%%%%%%%%%%%%%%%%%%%%%%%%%%%%
%%%%%%%%%%%%%%%%%%%%%%%%%%%%%%%%%%%%%%%%%%%%%%%%%%%%%%%%%%%%%%%%%%%%%%%%%%%%%%%
%%%%%%%%%%%%%%%%%%%%%%%%%%%%%%%%%%%%%%%%%%%%%%%%%%%%%%%%%%%%%%%%%%%%%%%%%%%%%%%
\begin{frame}[fragile]{Exercise 1: Typesetting}

\begin{block}{Typeset this in \LaTeX:
\footnote{\url{https://www.canada.ca/en/housing-infrastructure-communities/news/2025/03/protecting-the-chignecto-isthmus-against-climate-change.html}}}
New Brunswick and Nova Scotia committed to paying \$162.5 million each for a total of \$325 million to support the Chignecto Isthmus project (valued at \$650 million).

Each province is contributing 25\% of the total cost.
\end{block}
\vskip 2ex
\begin{center}
%\fbox{\href{\wlnewdoc{basics-exercise-1.tex}}{%
%Click to open this exercise in \wllogo{}}}
\end{center}

\begin{itemize}
\item Hint: watch out for characters with special meanings!
%\item Once you've tried,
%\fbox{\href{\wlnewdoc{basics-exercise-1-solution.tex}}{%
%click here to see my solution}}.
\end{itemize}
\end{frame}
%%%%%%%%%%%%%%%%%%%%%%%%%%%%%%%%%%%%%%%%%%%%%%%%%%%%%%%%%%%%%%%%%%%%%%%%%%%%%%%
%%%%%%%%%%%%%%%%%%%%%%%%%%%%%%%%%%%%%%%%%%%%%%%%%%%%%%%%%%%%%%%%%%%%%%%%%%%%%%%
%%%%%%%%%%%%%%%%%%%%%%%%%%%%%%%%%%%%%%%%%%%%%%%%%%%%%%%%%%%%%%%%%%%%%%%%%%%%%%%
\begin{frame}[fragile]{Concept: Packages}

\begin{itemize}
\item All of the commands and environments we've used so far are built into
\LaTeX.

\item \emph{Packages} are libraries of extra commands and environments. There
are thousands of freely available packages.

\item We have to load each of the packages we want to use with a
\cmdbs{usepackage} command in the \emph{preamble}.

\item Example: \bftt{amsmath} from the American Mathematical Society.
\item Example: \bftt{graphicx}
\begin{minted}[fontsize=\small,frame=single]{latex}
\documentclass{article}
\usepackage{graphicx}
\usepackage{amsmath}
\usepackage{biblatex} % preamble
\begin{document}
% now we can use commands from these packages! 
\end{document}
\end{minted}
\end{itemize}
\end{frame}

%%%%%%%%%%%%%%%%%%%%%%%%%%%%%%%%%%%%%%%%%%%%%%%%%%%%%%%%%%%%%%%%%%%%%%%%%%%%%%%
%%%%%%%%%%%%%%%%%%%%%%%%%%%%%%%%%%%%%%%%%%%%%%%%%%%%%%%%%%%%%%%%%%%%%%%%%%%%%%%
%%%%%%%%%%%%%%%%%%%%%%%%%%%%%%%%%%%%%%%%%%%%%%%%%%%%%%%%%%%%%%%%%%%%%%%%%%%%%%%
\begin{frame}[fragile]{Concept: Optional Arguments}
\begin{itemize}
\item We use square brackets \keystrokebftt{[} \keystrokebftt{]} for optional
arguments, instead of braces \keystrokebftt{\{} \keystrokebftt{\}}.
\item \cmdbs{includegraphics} accepts optional arguments that allow you to transform the
image when it is included. For example, \bftt{width=0.3\cmdbs{textwidth}} makes
the image take up 30\% of the width of the surrounding text (\cmdbs{textwidth}).
\item \cmdbs{documentclass} accepts optional arguments, too. Example:
\mint{latex}|\documentclass[12pt,twocolumn]{article}|
\vskip 3ex
makes the text bigger (12pt) and puts it into two columns.
\item Where do you find out about these? See the slides at the end of this
presentation for links to more information.
\end{itemize}
\end{frame}

%%%%%%%%%%%%%%%%%%%%%%%%%%%%%%%%%%%%%%%%%%%%%%%%%%%%%%%%%%%%%%%%%%%%%%%%%%%%%%%
%%%%%%%%%%%%%%%%%%%%%%%%%%%%%%%%%%%%%%%%%%%%%%%%%%%%%%%%%%%%%%%%%%%%%%%%%%%%%%%
%%%%%%%%%%%%%%%%%%%%%%%%%%%%%%%%%%%%%%%%%%%%%%%%%%%%%%%%%%%%%%%%%%%%%%%%%%%%%%%
\begin{frame}[fragile]{Concept: Environments}
\begin{itemize}
\item The \cmdbs{begin} and \cmdbs{end} commands are used to create many
different ``environments.'' Examples: document, equations, images.
\vskip 2ex

\item The \bftt{itemize} and \bftt{enumerate} environments generate lists.
\begin{exampletwouptiny}
\begin{itemize} % for bullet points
\item New Brunswick
\item Nova Scotia
\end{itemize}

\begin{enumerate} % for numbers
\item Wake up.
\item Find coffee!
\end{enumerate}
\end{exampletwouptiny}
\end{itemize}
\end{frame}
%%%%%%%%%%%%%%%%%%%%%%%%%%%%%%%%%%%%%%%%%%%%%%%%%%%%%%%%%%%%%%%%%%%%%%%%%%%%%%%
%%%%%%%%%%%%%%%%%%%%%%%%%%%%%%%%%%%%%%%%%%%%%%%%%%%%%%%%%%%%%%%%%%%%%%%%%%%%%%%
%%%%%%%%%%%%%%%%%%%%%%%%%%%%%%%%%%%%%%%%%%%%%%%%%%%%%%%%%%%%%%%%%%%%%%%%%%%%%%%
\begin{frame}[fragile]{Exercise 2: Add an image}

\begin{exampletwoup}
\begin{figure}
\includegraphics[
  width=0.5\textwidth]
  {chignecto_bridge.jpg}
\end{figure}
\end{exampletwoup}

\tiny{Image license: \href{https://www.flickr.com/photos/29442760@N00/54349740863/}{CC BY-SA 2.0, By A Disappearing Act}}
\end{frame}

%%%%%%%%%%%%%%%%%%%%%%%%%%%%%%%%%%%%%%%%%%%%%%%%%%%%%%%%%%%%%%%%%%%%%%%%%%%%%%%
%%%%%%%%%%%%%%%%%%%%%%%%%%%%%%%%%%%%%%%%%%%%%%%%%%%%%%%%%%%%%%%%%%%%%%%%%%%%%%%
%%%%%%%%%%%%%%%%%%%%%%%%%%%%%%%%%%%%%%%%%%%%%%%%%%%%%%%%%%%%%%%%%%%%%%%%%%%%%%%
\subsection{Typesetting Mathematics}
\begin{frame}[fragile]{\insertsubsection{}: Dollar Signs}
\begin{itemize}
\item Why are dollar signs \keystrokebftt{\$} special? We use them to mark mathematics in text.\\[1ex]
\begin{exampletwouptiny}
% not so good:
Let a and b be distinct positive
integers, and let c = a - b + 1.

% much better:
Let $a$ and $b$ be distinct positive
integers, and let $c = a - b + 1$.
\end{exampletwouptiny}
\item Always use dollar signs in pairs --- one to begin the mathematics, and one
to end it.
\item \LaTeX{} handles spacing automatically; it ignores your spaces.
\begin{exampletwouptiny}
Let $y=mx+b$ be \ldots

Let $y = m x + b$ be \ldots
\end{exampletwouptiny}
\end{itemize}
\end{frame}

%%%%%%%%%%%%%%%%%%%%%%%%%%%%%%%%%%%%%%%%%%%%%%%%%%%%%%%%%%%%%%%%%%%%%%%%%%%%%%%
%%%%%%%%%%%%%%%%%%%%%%%%%%%%%%%%%%%%%%%%%%%%%%%%%%%%%%%%%%%%%%%%%%%%%%%%%%%%%%%
%%%%%%%%%%%%%%%%%%%%%%%%%%%%%%%%%%%%%%%%%%%%%%%%%%%%%%%%%%%%%%%%%%%%%%%%%%%%%%%
\begin{frame}[fragile]{\insertsubsection{}: Notation}
\begin{itemize}
\item Use caret \keystrokebftt{\^} for superscripts and underscore \keystrokebftt{\_} for subscripts.
\begin{exampletwouptiny}
$y = c_2 x^2 + c_1 x + c_0$
\end{exampletwouptiny}
\vskip 2ex

\item Use curly braces \keystrokebftt{\{} \keystrokebftt{\}} to group
superscripts and subscripts.
\begin{exampletwouptiny}
$F_n = F_n-1 + F_n-2$     % oops!

$F_n = F_{n-1} + F_{n-2}$ % ok!
\end{exampletwouptiny}
\vskip 2ex

\item There are commands for Greek letters and common notation.
\begin{exampletwouptiny}
$\mu = A e^{Q/RT}$

$\Omega = \sum_{k=1}^{n} \omega_k$
\end{exampletwouptiny}
\end{itemize}
\end{frame}

%%%%%%%%%%%%%%%%%%%%%%%%%%%%%%%%%%%%%%%%%%%%%%%%%%%%%%%%%%%%%%%%%%%%%%%%%%%%%%%
%%%%%%%%%%%%%%%%%%%%%%%%%%%%%%%%%%%%%%%%%%%%%%%%%%%%%%%%%%%%%%%%%%%%%%%%%%%%%%%
%%%%%%%%%%%%%%%%%%%%%%%%%%%%%%%%%%%%%%%%%%%%%%%%%%%%%%%%%%%%%%%%%%%%%%%%%%%%%%%
\begin{frame}[fragile]{\insertsubsection{}: Displayed Equations}
\begin{itemize}
\item If it's big and scary, \emph{display} it on its own line using
\cmdbegin{equation} and \cmdend{equation}.\\[2ex]
\begin{exampletwouptiny}
The roots of a quadratic equation
are given by
\begin{equation}
x = \frac{-b \pm \sqrt{b^2 - 4ac}}
         {2a}
\end{equation}
where $a$, $b$ and $c$ are \ldots
\end{exampletwouptiny}
\vskip 1em
{\scriptsize Caution: \LaTeX{} mostly ignores your spaces in mathematics, but it
can't handle blank lines in equations --- don't put blank lines in your
mathematics.}
\end{itemize}
\end{frame}

%%%%%%%%%%%%%%%%%%%%%%%%%%%%%%%%%%%%%%%%%%%%%%%%%%%%%%%%%%%%%%%%%%%%%%%%%%%%%%%
%%%%%%%%%%%%%%%%%%%%%%%%%%%%%%%%%%%%%%%%%%%%%%%%%%%%%%%%%%%%%%%%%%%%%%%%%%%%%%%
%%%%%%%%%%%%%%%%%%%%%%%%%%%%%%%%%%%%%%%%%%%%%%%%%%%%%%%%%%%%%%%%%%%%%%%%%%%%%%%
\begin{frame}[fragile]{Concept: Environments}
\begin{itemize}
\item \bftt{equation} is an \emph{environment} --- a context.
\item A command can produce different output in different contexts.
\begin{exampletwouptiny}
We can write
$ \Omega = \sum_{k=1}^{n} \omega_k $
in text, or we can write
\begin{equation}
  \Omega = \sum_{k=1}^{n} \omega_k
\end{equation}
to display it.
\end{exampletwouptiny}
\vskip 2ex
\item Note how the $\Sigma$ is bigger in the \bftt{equation} environment, and
how the subscripts and superscripts change position, even though we used the
same commands.
\vskip 1em
{\scriptsize In fact, we could have written \bftt{\$...\$} as
\cmdbegin{math}\bftt{...}\cmdend{math}.}
\end{itemize}
\end{frame}
%%%%%%%%%%%%%%%%%%%%%%%%%%%%%%%%%%%%%%%%%%%%%%%%%%%%%%%%%%%%%%%%%%%%%%%%%%%%%%%
%%%%%%%%%%%%%%%%%%%%%%%%%%%%%%%%%%%%%%%%%%%%%%%%%%%%%%%%%%%%%%%%%%%%%%%%%%%%%%%
%%%%%%%%%%%%%%%%%%%%%%%%%%%%%%%%%%%%%%%%%%%%%%%%%%%%%%%%%%%%%%%%%%%%%%%%%%%%%%%
\begin{frame}[fragile]{\insertsubsection{}: Examples with \bftt{amsmath}}
\begin{itemize}
\item Use \bftt{equation*} (``equation-star'') for unnumbered equations.
\begin{exampletwouptiny}
\begin{equation*}
  \Omega = \sum_{k=1}^{n} \omega_k
\end{equation*}
\end{exampletwouptiny}
\item \LaTeX{} treats adjacent letters as variables multiplied together, which
is not always what you want. \bftt{amsmath} defines commands for many common
mathematical operators.
\begin{exampletwouptiny}
\begin{equation*} % bad!
 min_{x,y} (1-x)^2 + 100(y-x^2)^2
\end{equation*}
\begin{equation*} % good!
\min_{x,y}{(1-x)^2 + 100(y-x^2)^2}
\end{equation*}
\end{exampletwouptiny}
\item You can use \cmdbs{operatorname} for others.
\begin{exampletwouptiny}
\begin{equation*}
\beta_i =
\frac{\operatorname{Cov}(R_i, R_m)}
     {\operatorname{Var}(R_m)}
\end{equation*}
\end{exampletwouptiny}
\end{itemize}
\end{frame}

%%%%%%%%%%%%%%%%%%%%%%%%%%%%%%%%%%%%%%%%%%%%%%%%%%%%%%%%%%%%%%%%%%%%%%%%%%%%%%%
%%%%%%%%%%%%%%%%%%%%%%%%%%%%%%%%%%%%%%%%%%%%%%%%%%%%%%%%%%%%%%%%%%%%%%%%%%%%%%%
%%%%%%%%%%%%%%%%%%%%%%%%%%%%%%%%%%%%%%%%%%%%%%%%%%%%%%%%%%%%%%%%%%%%%%%%%%%%%%%
\begin{frame}[fragile]{\insertsubsection{}: Examples with \bftt{amsmath}}
\begin{itemize}{\small
\item Align a sequence of equations at the equals sign
\begin{align*}
(x+1)^3 &= (x+1)(x+1)(x+1) \\
        &= (x+1)(x^2 + 2x + 1) \\
        &= x^3 + 3x^2 + 3x + 1
\end{align*}
with the \bftt{align*} environment.

% for whatever reason, this doesn't play well with the twoup environment
\begin{minted}[fontsize=\small,frame=single]{latex}
\begin{align*}
(x+1)^3 &= (x+1)(x+1)(x+1) \\
        &= (x+1)(x^2 + 2x + 1) \\
        &= x^3 + 3x^2 + 3x + 1
\end{align*}
\end{minted}
\item An ampersand \keystrokebftt{\&} separates the left column (before the
$=$) from the right column (after the $=$).
\item A double backslash \keystrokebftt{\bs}\keystrokebftt{\bs} starts a new
line.
}\end{itemize}
\end{frame}


%%%%%%%%%%%%%%%%%%%%%%%%%%%%%%%%%%%%%%%%%%%%%%%%%%%%%%%%%%%%%%%%%%%%%%%%%%%%%%%
%%%%%%%%%%%%%%%%%%%%%%%%%%%%%%%%%%%%%%%%%%%%%%%%%%%%%%%%%%%%%%%%%%%%%%%%%%%%%%%
%%%%%%%%%%%%%%%%%%%%%%%%%%%%%%%%%%%%%%%%%%%%%%%%%%%%%%%%%%%%%%%%%%%%%%%%%%%%%%%
\begin{frame}[fragile]{Exercise 3: Typesetting Mathematics}

\begin{block}{Typeset this in \LaTeX:}
Let $X_1, X_2, \ldots, X_n$ be a sequence of independent and identically
distributed random variables with $\operatorname{E}[X_i] = \mu$ and
$\operatorname{Var}[X_i] = \sigma^2 < \infty$, and let
\begin{equation*}
S_n = \frac{1}{n}\sum_{i=1}^{n} X_i
\end{equation*}
denote their mean. Then as $n$ approaches infinity, the random variables
$\sqrt{n}(S_n - \mu)$ converge in distribution to a normal $N(0, \sigma^2)$.
\end{block}
\vskip 2ex
\begin{center}
\fbox{\href{\wlnewdoc{basics-exercise-2.tex}}{%
Click to open this exercise in \wllogo{}}}
\end{center}
\begin{itemize}
\item Hint: the command for $\infty$ is \cmdbs{infty}.
\item Once you've tried,
\fbox{\href{\wlnewdoc{basics-exercise-2-solution.tex}}{%
click here to see my solution}}.
\end{itemize}
\end{frame}

%%%%%%%%%%%%%%%%%%%%%%%%%%%%%%%%%%%%%%%%%%%%%%%%%%%%%%%%%%%%%%%%%%%%%%%%%%%%%%%
%%%%%%%%%%%%%%%%%%%%%%%%%%%%%%%%%%%%%%%%%%%%%%%%%%%%%%%%%%%%%%%%%%%%%%%%%%%%%%%
%%%%%%%%%%%%%%%%%%%%%%%%%%%%%%%%%%%%%%%%%%%%%%%%%%%%%%%%%%%%%%%%%%%%%%%%%%%%%%%

\section{Formatting Document}
\subsection{Preamble}
\begin{frame}[fragile]{\insertsection : \insertsubsection}
We've mentioned the preamble a few times. What is it? 
Everything above \cmdbs{begin{document}}
\begin{minted}[fontsize=\small,frame=single]{latex}
\documentclass{article}
\usepackage{graphicx}
\usepackage{amsmath}
\usepackage{biblatex}
\usepackage[utf8]{inputenc} % add this one

\title{2025 LaTeX Workshop} % for title page
\author{Elizabeth Stregger}
\date{March 27, 2025}
\end{minted}
\end{frame}
%%%%%%%%%%%%%%%%%%%%%%%%%%%%%%%%%%%%%%%%%%%%%%%%%%%%%%%%%%%%%%%%%%%%%%%%%%%%%%%
%%%%%%%%%%%%%%%%%%%%%%%%%%%%%%%%%%%%%%%%%%%%%%%%%%%%%%%%%%%%%%%%%%%%%%%%%%%%%%%
%%%%%%%%%%%%%%%%%%%%%%%%%%%%%%%%%%%%%%%%%%%%%%%%%%%%%%%%%%%%%%%%%%%%%%%%%%%%%%%
\subsection{Sections}
\begin{frame}[fragile]{\insertsection : \insertsubsection}
\begin{itemize}{\small
\item Use \cmdbs{section} and \cmdbs{subsection}.
\item Can you guess what \cmdbs{section*} and \cmdbs{subsection*} do?
}\end{itemize}
\begin{minipage}{0.55\linewidth}
\inputminted[fontsize=\scriptsize,frame=single,resetmargins]{latex}%
  {structure-sections.tex}
\end{minipage}
\begin{minipage}{0.35\linewidth}
\includegraphics[width=\textwidth,clip,trim=1.5in 6in 4in 1in]{structure-sections.pdf}
\end{minipage}
\end{frame}
%%%%%%%%%%%%%%%%%%%%%%%%%%%%%%%%%%%%%%%%%%%%%%%%%%%%%%%%%%%%%%%%%%%%%%%%%%%%%%%
%%%%%%%%%%%%%%%%%%%%%%%%%%%%%%%%%%%%%%%%%%%%%%%%%%%%%%%%%%%%%%%%%%%%%%%%%%%%%%%
%%%%%%%%%%%%%%%%%%%%%%%%%%%%%%%%%%%%%%%%%%%%%%%%%%%%%%%%%%%%%%%%%%%%%%%%%%%%%%%
\subsection{Title, Author, Table of Contents}
\begin{frame}[fragile]{\insertsection : \insertsubsection}
\begin{itemize}
\item Use the \cmdbs{maketitle} command to add the title section
\item Use the \cmdbs{tableofcontents} command to generate a table of contents
from the \cmdbs{section} commands.
\end{itemize}
\begin{minipage}{0.55\linewidth}
\inputminted[fontsize=\scriptsize,frame=single,resetmargins]{latex}%
  {toc.tex}
\end{minipage}
\begin{minipage}{0.35\linewidth}
\includegraphics[width=1.5\textwidth]{toc.pdf}
\end{minipage}
\end{frame}
%%%%%%%%%%%%%%%%%%%%%%%%%%%%%%%%%%%%%%%%%%%%%%%%%%%%%%%%%%%%%%%%%%%%%%%%%%%%%%%
%%%%%%%%%%%%%%%%%%%%%%%%%%%%%%%%%%%%%%%%%%%%%%%%%%%%%%%%%%%%%%%%%%%%%%%%%%%%%%%
%%%%%%%%%%%%%%%%%%%%%%%%%%%%%%%%%%%%%%%%%%%%%%%%%%%%%%%%%%%%%%%%%%%%%%%%%%%%%%%
\section{Last but Not Least}
\subsection{Citations}
\begin{frame}[fragile]{\insertsection : \insertsubsection}
\begin{itemize}
\item Citations for this workshop are included in  \textbf{chignecto.bib}.
\item This file was created by exporting a Zotero collection in BibLateX format with UTF-8 encoding.
\begin{minted}[fontsize=\small,frame=single]{latex}
\documentclass{article}
\usepackage{biblatex}
\usepackage[utf8]{inputenc}

\addbibresource{chignecto.bib} % add to preamble

% use the \cite{} command to insert citations
% add \printbibliography before \end{document}
\end{minted}
\end{itemize}
\end{frame}
%%%%%%%%%%%%%%%%%%%%%%%%%%%%%%%%%%%%%%%%%%%%%%%%%%%%%%%%%%%%%%%%%%%%%%%%%%%%%%%
%%%%%%%%%%%%%%%%%%%%%%%%%%%%%%%%%%%%%%%%%%%%%%%%%%%%%%%%%%%%%%%%%%%%%%%%%%%%%%%
%%%%%%%%%%%%%%%%%%%%%%%%%%%%%%%%%%%%%%%%%%%%%%%%%%%%%%%%%%%%%%%%%%%%%%%%%%%%%%%
\begin{frame}[fragile]{\insertsection : \insertsubsection}
\includegraphics[width=1\textwidth]{minimum-bib.pdf}
\end{frame}
%%%%%%%%%%%%%%%%%%%%%%%%%%%%%%%%%%%%%%%%%%%%%%%%%%%%%%%%%%%%%%%%%%%%%%%%%%%%%%%
%%%%%%%%%%%%%%%%%%%%%%%%%%%%%%%%%%%%%%%%%%%%%%%%%%%%%%%%%%%%%%%%%%%%%%%%%%%%%%%
%%%%%%%%%%%%%%%%%%%%%%%%%%%%%%%%%%%%%%%%%%%%%%%%%%%%%%%%%%%%%%%%%%%%%%%%%%%%%%%
\begin{frame}{End of Part 1}
\begin{itemize}
\item Congrats! You've already learned how to \ldots
\begin{itemize}
\item Typeset text and mathematics in \LaTeX.
\item Add images and bibliography.
\item Format your document.
\end{itemize}
\item That's amazing!
\item We have an extra challenge for you: typesetting a full document.
\begin{itemize}
\item The web-formatted text is at the URL: \href{https://www.canada.ca/en/housing-infrastructure-communities/news/2025/03/protecting-the-chignecto-isthmus-against-climate-change.html}{Chignecto Isthmus funding announcement}
\item Our solution is the file \textbf{chignecto-announcement.tex}
\end{itemize}
\end{itemize}
\end{frame}

\end{document}
