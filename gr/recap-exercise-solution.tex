\documentclass[12pt]{article}

\usepackage{url}
\usepackage[greek]{babel}
\usepackage[utf8]{inputenc}

\title{Δέκα μυστικά για να κάνετε μια καλή επιστημονική ομιλία}
\author{Εσείς}

\begin{document}
\maketitle

\section{Εισαγωγή}

Το κείμενο για αυτήν την άσκηση είναι μια σημαντικά συντομευμένη και ελαφρώς τροποποιημένη εκδοχή του εξαιρετικού ομώνυμου άρθρου των \textlatin{Mark Schoeberl} και \textlatin{Brian Toon}:
\textlatin{ \url{http://www.cgd.ucar.edu/cms/agu/scientific_talk.html}}

\section{Τα Μυστικά}

Έχω συντάξει αυτήν την προσωπική λίστα με τα «Μυστικά» από την ακρόαση αποτελεσματικών και αναποτελεσματικών ομιλητών. Δεν προσποιούμαι ότι αυτή η λίστα είναι περιεκτική - είμαι σίγουρος ότι υπάρχουν πράγματα που έχω αφήσει έξω. Όμως, η λίστα μου καλύπτει πιθανώς περίπου το 90\% αυτών που πρέπει να γνωρίζετε και να κάνετε.

\begin{enumerate}

\item Προετοιμάστε το υλικό σας προσεκτικά και λογικά. Πείτε μια ιστορία.

\item Εξασκήστε την ομιλία σας. Δεν υπάρχει δικαιολογία για έλλειψη προετοιμασίας.

\item Μην βάζετε πολύ υλικό. Οι καλοί ομιλητές θα έχουν ένα ή δύο κεντρικά σημεία και θα μένουν σε αυτό το υλικό.

\item Αποφύγετε τις εξισώσεις. Λέγεται ότι για κάθε εξίσωση στην ομιλία σας, ο αριθμός των ατόμων που θα την καταλάβουν θα μειώνεται στο μισό. Δηλαδή, αν αφήσουμε q να είναι ο αριθμός των εξισώσεων στην ομιλία σας και n ο αριθμός των ατόμων που καταλαβαίνουν την ομιλία σας, ισχύει ότι
\begin{equation}
n = \gamma \left( \frac{1}{2} \right)^q
\end{equation}
όπου το $\gamma$ είναι σταθερά αναλογικότητας.

\item Έχετε μόνο μερικά συμπεράσματα. Οι άνθρωποι δεν μπορούν να θυμηθούν περισσότερα από μερικά πράγματα από μια ομιλία, ειδικά αν ακούνε πολλές ομιλίες σε μεγάλες συνεδριάσεις.

\item Μιλήστε στο κοινό, όχι στην οθόνη. Ένα από τα πιο συνηθισμένα προβλήματα που βλέπω είναι ότι ο ομιλητής θα μιλήσει στην οθόνη του \textlatin{viewgraph}.

\item Αποφύγετε να κάνετε ήχους που αποσπούν την προσοχή. Προσπαθήστε να αποφύγετε τα «Αμμμ» ή «Αααα» μεταξύ των προτάσεων.

\item Δώστε έμφαση στα γραφικά σας. Ακολουθεί μια λίστα με συμβουλές για καλύτερα γραφικά:

\begin{itemize}
\item Χρησιμοποιήστε μεγάλα γράμματα.

\item Διατηρήστε τα γραφικά απλά. Μην εμφανίζετε γραφήματα που δεν θα χρειαστείτε.

\item Χρησιμοποιήστε χρώμα.

\end{itemize}

\item Να είστε ευγενικοί στο να λαμβάνετε ερωτήσεις.

\item Χρησιμοποιήστε χιούμορ αν είναι δυνατόν. Πάντα εκπλήσσομαι πώς ακόμα και ένα  αστείο θα προκαλέσει ένα καλό γέλιο σε μια επιστημονική ομιλία.

\end{enumerate}

\end{document}
